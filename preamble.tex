\usepackage{amsthm,amsmath, amsfonts}
%\usepackage[ngerman]{babel}
\usepackage{tcolorbox}
\usepackage{hyperref, cleveref}
\usepackage[dvipsnames]{xcolor}
\usepackage{thmtools}
\usepackage{enumitem}
%\usepackage{marginnote}
\usepackage{todonotes}
\usepackage{titling}
\usepackage{geometry}

\newcounter{aufgabe}

%%%%%%%%%%%%%%%%%%
%% Fancy Header %% 
%%%%%%%%%%%%%%%%%%
%May be uncommented if not liked
\usepackage{fancyhdr}
\fancyhead[L]{\textbf{\thetitle}}
\fancyhead[R]{\theauthor}
\setlength{\headheight}{14pt}
% reversing title drop from margin
\setlength{\droptitle}{-4em} 
\setlength{\parindent}{0pt} 

\newcommand{\mycomment}[1]{}

\newcommand{\innertopmargin}{1.2\topskip}
\newcommand{\innerbottommargin}{0.5\topskip}


%%%%%%%%%%%%
%% Margin %%
%%%%%%%%%%%%
\geometry{
	left=1in,
	right=2.5in,
	top=2cm,
	bottom=3cm,
	marginparwidth=3cm,
	marginparsep=0.8cm,
	a4paper
}

%%%%%%%%%%%
%% Fonts %%
%%%%%%%%%%%
\usepackage{libertinus}
\usepackage[libertine]{newtxmath}

%%%%%%%%%%%%
%% Colors %%
%%%%%%%%%%%%
\definecolor{astral}{RGB}{3,57,108}
\definecolor{dblue}{RGB}{1,31,75}


%%%%%%%%%%%%%
%% General %%
%%%%%%%%%%%%%
\newcommand{\C}{\mathbb{C}}
\newcommand{\R}{\mathbb{R}}
\newcommand{\Q}{\mathbb{Q}}
\newcommand{\N}{\mathbb{N}}
\newcommand{\Z}{\mathbb{Z}}
\newcommand{\HH}{\mathbb{H}}
\newcommand{\ohne}{\setminus}
\newcommand{\card}[1]{\operatorname{\#}\left(#1 \right)}
\renewcommand{\iff}{\Leftrightarrow}
\renewcommand{\implies}{\Rightarrow}
\renewcommand{\tilde}{\widetilde}
\renewcommand{\bar}{\overline}
\newcommand{\aufz}[2]{#1_1, \dots, #1_{#2}}
\renewcommand{\P}{\mathcal{P}}
\newcommand{\Ps}{\mathscr{P}}
\newcommand{\vphi}{\varphi}
\renewcommand{\phi}{\varphi}
\newcommand{\norm}[1]{\left\lVert #1 \right\rVert}
\newcommand{\abs}[1]{\left\lvert #1 \right\rvert}
\newcommand{\cls}[1]{\left[ #1 \right]}
\newcommand{\floor}[1]{\left\lfloor #1 \right\rfloor}
\DeclareMathOperator{\pr}{pr}
\DeclareMathOperator{\re}{Re}
\DeclareMathOperator{\imag}{Im}


%%%%%%%%%%%%%%%%%%%%%
%% Category Theory %%
%%%%%%%%%%%%%%%%%%%%%
\DeclareMathOperator{\Ob}{Ob}
\newcommand{\Cat}{\mathcal{C}}
\DeclareMathOperator{\Mor}{Mor}
\DeclareMathOperator{\Set}{\mathbf{Set}}
\DeclareMathOperator{\CTop}{\mathbf{Top}}
\DeclareMathOperator{\HoTop}{\mathbf{HoTop}}
\DeclareMathOperator{\CVec}{\mathbf{Vec}}
\DeclareMathOperator{\Grp}{\mathbf{Grp}}
\DeclareMathOperator{\AbGrp}{\mathbf{Ab}}
\DeclareMathOperator{\Ring}{\mathbf{Ring}}
\DeclareMathOperator{\RMod}{\mathbf{R-Mod}}
\DeclareMathOperator{\AMod}{\mathbf{A-Mod}}


%%%%%%%%%%%%%
%% Algebra %%
%%%%%%%%%%%%%
\DeclareMathOperator{\Grad}{Grad}
\DeclareMathOperator{\minpol}{minpol}
\DeclareMathOperator{\charpol}{charpol}

\DeclareMathOperator{\id}{id}
\DeclareMathOperator{\Id}{Id}
\DeclareMathOperator{\ord}{ord}
\DeclareMathOperator{\sgn}{sgn}
\newcommand{\nsub}{\vartriangleright}
\newcommand{\nsubl}{\vartriangleleft}

\DeclareMathOperator{\ggT}{ggT}
\DeclareMathOperator{\kgv}{kgV}
\DeclareMathOperator{\im}{im}
\DeclareMathOperator{\Bild}{Bild}
\DeclareMathOperator{\Aut}{Aut}
\DeclareMathOperator{\End}{End}
\DeclareMathOperator{\Lin}{\mathcal{L}}
\DeclareMathOperator{\Bij}{Bij}
\DeclareMathOperator{\Abb}{Abb}
\DeclareMathOperator{\Map}{Map}
\DeclareMathOperator{\Mod}{Mod}
\DeclareMathOperator{\Hom}{Hom}

\DeclareMathOperator{\Gal}{Gal}
\DeclareMathOperator{\F}{\mathbb{F}}
\DeclareMathOperator{\Cl}{Cl}
\DeclareMathOperator{\Spec}{Spec}
\DeclareMathOperator{\Lok}{Lok}
\newcommand{\OK}{\mathcal{O}_K}
\newcommand{\OL}{\mathcal{O}_L}
\newcommand{\legendre}[2]{\left(\dfrac{#1}{#2}\right)}
\renewcommand{\O}{\mathcal{O}}

\renewcommand{\AA}{\mathbb{A}}
\newcommand{\PP}{\mathbb{P}}
\newcommand{\II}{\mathbb{I}}
\newcommand{\KK}{\mathbb{K}}

\DeclareMathOperator{\Ker}{Ker}
\DeclareMathOperator{\Span}{Span}
\newcommand{\iso}{\cong}
\newcommand{\B}{\mathcal{B}}
\newcommand{\einmat}{\mathbb{E}}
\newcommand{\bform}[2]{\left\langle #1, #2 \right\rangle}
\DeclareMathOperator{\GL}{GL}
\DeclareMathOperator{\SL}{SL}
\DeclareMathOperator{\Orth}{O}
\DeclareMathOperator{\SO}{SO}
\DeclareMathOperator{\rank}{rank}
\DeclareMathOperator{\disc}{disc}
\DeclareMathOperator{\tr}{tr}
\DeclareMathOperator{\Tr}{Tr}

\newcommand{\pp}{\mathfrak{p}}
\renewcommand{\aa}{\mathfrak{a}}
\newcommand{\bb}{\mathfrak{b}}
\newcommand{\qq}{\mathfrak{q}}
\newcommand{\cc}{\mathfrak{c}}
\newcommand{\mm}{\mathfrak{m}}
\newcommand{\zpz}{\Z / p\Z}
\newcommand{\zmz}{\Z / m\Z}
\newcommand{\znz}{\Z / n\Z}
\newcommand{\zNz}{\Z / N\Z}
\newcommand{\calt}{\mathcal{T}}
\newcommand{\caln}{\mathcal{N}}
\newcommand{\Zi}{\Z[i]}
\newcommand{\Qi}{\Q[i]}
\newcommand{\zmod}[1]{\Z / #1\Z}
\newcommand{\kxn}[1]{k[x_1, \dots, x_{#1}]}
\newcommand{\kx}{k[x]}
\newcommand{\kyn}[1]{k[y_1, \dots, y_{#1}]}
\newcommand{\ky}{k[y]}
\newcommand{\gen}[1]{\langle #1 \rangle}
\newcommand{\epim}{\twoheadrightarrow}
\newcommand{\monom}{\hookrightarrow}

\DeclareMathOperator{\Zgez}{\Z_{\geq 0}}
\DeclareMathOperator{\Zgz}{\Z_{> 0}}
\DeclareMathOperator{\Zlez}{\Z_{\leq 0}}
\DeclareMathOperator{\Zlz}{\Z_{< 0}}

\DeclareMathOperator{\Qgez}{\Q_{\geq 0}}
\DeclareMathOperator{\Qgz}{\Q_{> 0}}
\DeclareMathOperator{\Qlez}{\Q_{\leq 0}}
\DeclareMathOperator{\Qlz}{\Q_{< 0}}


%%%%%%%%%%%%%%
%% Analysis %%
%%%%%%%%%%%%%%
\newcommand{\eps}{\varepsilon}
\newcommand{\veps}{\epsilon}
\DeclareMathOperator{\holo}{\mathcal{O}}
\DeclareMathOperator{\vol}{vol}
\DeclareMathOperator{\Rgez}{\R_{\geq 0}}
\DeclareMathOperator{\Rgz}{\R_{> 0}}
\DeclareMathOperator{\Rlez}{\R_{\leq 0}}
\DeclareMathOperator{\Rlz}{\R_{< 0}}

%%%%%%%%%%%%%%
%% Theorems %%
%%%%%%%%%%%%%%
%% Boxed Theorems %%
\mycomment{ %% Comented out because not in use, replaced by mdframed
\tcbuselibrary{theorems}
\newtcbtheorem[number within=section, crefname={\mathrm{Satz}}{Satz}]{satz}{Satz}
{before skip=2em, after skip=2em, colback=red!5!white, colframe=red!50!black}{S}
\newtcbtheorem[number within=section, crefname={\mathrm{Theorem}}{Theorem}]{thm}{Theorem}
{before skip=2em, after skip=2em, colback=red!5!white, colframe=red!50!black}{S}
\newtcbtheorem[number within=section, crefname={\mathrm{Korollar}}{Korollar}]{kor}
{Korollar}{before skip=2em, after skip=2em}{K}
\newtcbtheorem[number within=section, crefname={\mathrm{Corollary}}{Corollary}]{cor}
{Corollary}{before skip=2em, after skip=2em}{C}
\newtcbtheorem[number within=section, crefname={\mathrm{Definition}}{Definition}]{Def}
{Definition}{before skip=2em, after skip=2em, colback=green!5!white, colframe=green!50!black}{D}
\newtcbtheorem[number within=section, crefname={\mathrm{Hilfssatz}}{Hilfssatz}]{hilfs}
{Hilfssatz}{before skip=2em, after skip=2em}{HS}
\newtcbtheorem[number within=section, crefname={\mathrm{Lemma}}{Lemma}]{Lemma}{Lemma}
{before skip=2em, after skip=2em}{L}
\newtcbtheorem[number within=section, crefname={\mathrm{Folgerung}}{Folgerung}]{folg}
{Folgerung}{before skip=2em, after skip=2em}{FOLG}
\newtcbtheorem[number within=section, crefname={\mathrm{Aufgabe}}{Aufgabe}]{auf}{Aufgabe}
{before skip=2em, after skip=2em, colback=blue!5!white, colframe=blue!50!black}{A}
\newtcbtheorem[number within=section, crefname={\mathrm{Proposition}}{Proposition}]
{prop}{Proposition}{before skip=2em, after skip=2em, colback=yellow!5!white, colframe=brown!50!black}{P}
}

%% Normal Theorems %%
\declaretheoremstyle[
	headfont=\normalfont,
	spaceabove=1em,
	spacebelow=1em
]{beisp}
\declaretheoremstyle[
	headfont=\color{purple}\normalfont,
	spaceabove=1em,
	spacebelow=1em
]{beme}
\declaretheoremstyle[
	headfont=\color{astral}\normalfont\bfseries,
	spaceabove=1em,
	spacebelow=1em
]{aufg}


\declaretheorem[
	style=beisp,
	name=Beispiel,
	numberwithin=section
]{bsp}
\declaretheorem[
	style=beisp,
	name=Example,
	numberwithin=section
]{ex}
\declaretheorem[
	style=beisp,
	name=Remark,
	numberwithin=section
]{rem}
\declaretheorem[
	style=beme,
	name=Bemerkung,
	numberwithin=section
]{bem}
\declaretheorem[
	style=aufg,
	name=Aufgabe,
]{auf}
\declaretheorem[
	style=aufg,
	name=Exercise,
]{exe}


%%%%%%%%%%%%%%%%%%%%%%%
%% mdframed Theorems %%
%%%%%%%%%%%%%%%%%%%%%%%


\usepackage[framemethod=TikZ]{mdframed}
\mdfsetup{skipabove=1em,skipbelow=0em}
\theoremstyle{definition}


\declaretheoremstyle[
    headfont=\bfseries\sffamily\color{ForestGreen!70!black}, bodyfont=\normalfont,
    mdframed={
        linewidth=2pt,
        rightline=false, topline=false, bottomline=false,
        linecolor=ForestGreen, backgroundcolor=ForestGreen!5,
		innertopmargin = \innertopmargin,
		innerbottommargin = \innerbottommargin,
    }
]{thmgreenbox}

\declaretheoremstyle[
    headfont=\bfseries\sffamily\color{NavyBlue!70!black}, bodyfont=\normalfont,
    mdframed={
        linewidth=2pt,
        rightline=false, topline=false, bottomline=false,
        linecolor=NavyBlue, backgroundcolor=NavyBlue!5,
		innertopmargin = \innertopmargin,
		innerbottommargin = \innerbottommargin,
    }
]{thmbluebox}

\declaretheoremstyle[
    headfont=\bfseries\sffamily\color{NavyBlue!70!black}, bodyfont=\normalfont,
    mdframed={
        linewidth=2pt,
        rightline=false, topline=false, bottomline=false,
        linecolor=NavyBlue,
		innertopmargin = \innertopmargin,
		innerbottommargin = \innerbottommargin,
    }
]{thmblueline}

\declaretheoremstyle[
    headfont=\bfseries\sffamily\color{RawSienna!70!black}, bodyfont=\normalfont,
    mdframed={
        linewidth=2pt,
        rightline=false, topline=false, bottomline=false,
        linecolor=RawSienna, backgroundcolor=RawSienna!5,
		innertopmargin = \innertopmargin,
		innerbottommargin = \innerbottommargin,
    }
]{thmredbox}

\declaretheoremstyle[
    headfont=\bfseries\sffamily\color{RawSienna!70!black}, bodyfont=\normalfont,
    numbered=no,
    mdframed={
        linewidth=2pt,
        rightline=false, topline=false, bottomline=false,
        linecolor=RawSienna, backgroundcolor=RawSienna!1,
		innertopmargin = \innertopmargin,
		innerbottommargin = \innerbottommargin,
    },
    qed=\qedsymbol
]{thmproofbox}

\declaretheoremstyle[
    headfont=\bfseries\sffamily\color{NavyBlue!70!black}, bodyfont=\normalfont,
    numbered=no,
    mdframed={
        linewidth=2pt,
        rightline=false, topline=false, bottomline=false,
        linecolor=NavyBlue, backgroundcolor=NavyBlue!1,
		innertopmargin = \innertopmargin,
		innerbottommargin = \innerbottommargin,
    },
]{thmexplanationbox}


\declaretheorem[name=Theorem, style=thmredbox]{Theorem}
\declaretheorem[name=Proof, style=thmproofbox]{tempproof}
\declaretheorem[name=Beweis, style=thmproofbox]{tempbew}
\declaretheorem[name=Proof, style=thmexplanationbox]{remproof}
\declaretheorem[name=Proof, style=thmexplanationbox]{bembew}
\newenvironment{Proof}[1][\proofname]{\vspace{-1em}\begin{tempproof}}{\end{tempproof}}
\newenvironment{Beweis}[1][\proofname]{\vspace{-1em}\begin{tempbew}}{\end{tempbew}}
\newenvironment{RemProof}[1][\proofname]{\vspace{-1em}\begin{remproof}}{\end{remproof}}
\newenvironment{BemBeweis}[1][\proofname]{\vspace{-1em}\begin{bembew}}{\end{bembew}}

\declaretheorem[name=Lemma, style=thmredbox]{Lemma}
\declaretheorem[name=Proposition, style=thmredbox]{Proposition}

\declaretheorem[name=Remark, style=thmbluebox]{Remark}
\declaretheorem[name=Bemerkung, style=thmbluebox]{Bemerkung}

\declaretheorem[name=Example, style=thmbluebox]{Example}
\declaretheorem[name=Beispiel, style=thmbluebox]{Beispiel}

\declaretheorem[name=Corollary, style=thmredbox]{Corollary}
\declaretheorem[name=Korollar, style=thmredbox]{Korollar}
\declaretheorem[name=Folgerung, style=thmredbox]{Folgerung}

\declaretheorem[name=Satz, style=thmredbox]{Satz}


