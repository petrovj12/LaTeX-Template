\usepackage{amsthm,amsmath,amssymb, amsfonts}
\usepackage[ngerman]{babel}
\usepackage{tcolorbox}
\usepackage{hyperref, cleveref}
\usepackage{xcolor, thmtools}
\usepackage{mathabx}
\usepackage{enumitem}
\usepackage{geometry}

\newcounter{aufgabe}

\geometry{
	left=1in,
	right=2.5in,
	top=2cm,
	bottom=3cm,
	a4paper
}

%--Colors--
\definecolor{astral}{RGB}{3,57,108}
\definecolor{dblue}{RGB}{1,31,75}


%--General--
\newcommand{\C}{\mathbb{C}}
\newcommand{\R}{\mathbb{R}}
\newcommand{\Q}{\mathbb{Q}}
\newcommand{\N}{\mathbb{N}}
\newcommand{\Z}{\mathbb{Z}}
\newcommand{\card}[1]{\operatorname{\#}\left(#1 \right)}
\renewcommand{\iff}{\Leftrightarrow}
\renewcommand{\implies}{\Rightarrow}
\renewcommand{\tilde}{\widetilde}
\renewcommand{\bar}{\overline}
\newcommand{\aufz}[2]{#1_1, \dots, #1_{#2}}

%--Algebra--

\DeclareMathOperator{\Grad}{Grad}
\DeclareMathOperator{\minpol}{minpol}
\DeclareMathOperator{\charpol}{charpol}
\DeclareMathOperator{\ord}{ord}
\DeclareMathOperator{\im}{im}
\DeclareMathOperator{\Bild}{Bild}
\DeclareMathOperator{\ggT}{ggT}
\DeclareMathOperator{\kgV}{kgV}
\DeclareMathOperator{\Gal}{Gal}
\DeclareMathOperator{\Aut}{Aut}
\DeclareMathOperator{\Bij}{Bij}
\DeclareMathOperator{\Abb}{Abb}
\DeclareMathOperator{\galf}{\mathbb{F}}
\DeclareMathOperator{\id}{id}
\DeclareMathOperator{\Ker}{Ker}
\DeclareMathOperator{\Span}{Span}
\DeclareMathOperator{\Spec}{Spec}
\DeclareMathOperator{\Lok}{Lok}
\DeclareMathOperator{\einmat}{\mathbb{E}}
\newcommand{\pp}{\mathfrak{p}}
\newcommand{\zpz}{\Z / p\Z}
\newcommand{\zmz}{\Z / m\Z}
\newcommand{\znz}{\Z / n\Z}
\newcommand{\zNz}{\Z / N\Z}
\newcommand{\zmod}[1]{\Z / #1\Z}
\newcommand{\kxn}[1]{k[x_1, \dots, x_{#1}]}
\newcommand{\kx}{k[x]}
\newcommand{\kyn}[1]{k[y_1, \dots, y_{#1}]}
\newcommand{\ky}{k[y]}
\newcommand{\epim}{\twoheadrightarrow}
\newcommand{\monom}{\hookrightarrow}
\newcommand{\nsub}{\vartriangleright}
\newcommand{\nsubl}{\vartriangleleft}

%--Analysis--

\newcommand{\eps}{\epsilon}
\newcommand{\veps}{\varepsilon}
\newcommand{\vphi}{\varphi}
\DeclareMathOperator{\re}{Re}
\DeclareMathOperator{\imag}{Im}
\DeclareMathOperator{\holo}{\mathcal{O}}

%--Theorems
\tcbuselibrary{theorems}
\newtcbtheorem[number within=section, crefname={\mathrm{Satz}}{Satz}]{satz}{Satz}
{before skip=2em, after skip=2em, colback=red!5!white, colframe=red!50!black}{S}
\newtcbtheorem[number within=section, crefname={\mathrm{Korollar}}{Korollar}]{kor}
{Korollar}{before skip=2em, after skip=2em}{K}
\newtcbtheorem[number within=section, crefname={\mathrm{Definition}}{Definition}]{Def}
{Definition}{before skip=2em, after skip=2em, colback=green!5!white, colframe=green!50!black}{D}
\newtcbtheorem[number within=section, crefname={\mathrm{Hilfssatz}}{Hilfssatz}]{hilfs}
{Hilfssatz}{before skip=2em, after skip=2em}{HS}
\newtcbtheorem[number within=section, crefname={\mathrm{Lemma}}{Lemma}]{Lemma}{Lemma}
{before skip=2em, after skip=2em}{L}
\newtcbtheorem[number within=section, crefname={\mathrm{Folgerung}}{Folgerung}]{folg}
{Folgerung}{before skip=2em, after skip=2em}{FOLG}
\newtcbtheorem[number within=section, crefname={\mathrm{Aufgabe}}{Aufgabe}]{auf}{Aufgabe}{before skip=2em, after skip=2em, colback=blue!5!white, colframe=blue!50!black}{A}
\newtcbtheorem[number within=section, crefname={\mathrm{Proposition}}{Proposition}]
{prop}{Proposition}{before skip=2em, after skip=2em, colback=yellow!5!white, colframe=brown!50!black}{P}

\declaretheoremstyle[
	headfont=\color{blue}\normalfont,
	spaceabove=1em,
	spacebelow=1em
]{beisp}
\declaretheoremstyle[
	headfont=\color{purple}\normalfont,
	spaceabove=1em,
	spacebelow=1em
]{beme}
\declaretheoremstyle[
	headfont=\color{astral}\normalfont\bfseries,
	spaceabove=1em,
	spacebelow=1em
]{aufg}
\declaretheorem[
	style=beisp,
	name=Beispiel,
	numberwithin=section
]{bsp}
\declaretheorem[
	style=beme,
	name=Bemerkung,
	numberwithin=section
]{bem}
\declaretheorem[
	style=aufg,
	name=Aufgabe,
]{Aufgabe}



